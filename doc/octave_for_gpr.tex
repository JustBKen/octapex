\documentclass[a4paper,12pt]{article}
\usepackage{ngerman}
\usepackage{soul}
\usepackage{mathtools}
\usepackage{amssymb,amsmath,amsfonts}
\usepackage[utf8]{inputenc}
\usepackage{graphicx}
\usepackage{geometry}
\usepackage{float}
\usepackage[autostyle=true,german=quotes]{csquotes}
\usepackage{hyperref}
\usepackage{fancyhdr}
\usepackage{gensymb}
\usepackage{units}
\usepackage{hhline}
\usepackage{color}
\usepackage[export]{adjustbox}
\usepackage[nottoc,numbib]{tocbibind}
\usepackage{titling}
\usepackage{enumitem}
\usepackage{wrapfig}
\usepackage[ref]{cite}
\usepackage[font=small,labelfont=bf]{caption}



\geometry{a4paper, left=20mm, right=20mm, top=20mm, bottom=20mm}
\definecolor{pantone294}{cmyk}{1,0.6,0,0.2}
\hypersetup{colorlinks, citecolor=black, filecolor=black, linkcolor=black, urlcolor=black }

\title{Octave für das Grundpraktikum}
\author{Anton Douginets}
\date{\today}
\pagestyle{plain}
\rhead{}
\lfoot{Humboldt-Universität zu Berlin}
\rfoot{octapex}


\begin{document}
	%\newgeometry{left=14mm, right=13.5mm, top=13.5mm, bottom=30mm}
	\begin{titlepage}
		\thispagestyle{empty}
		\begin{figure}
			\includegraphics[width=31.5mm,right]{/home/anton/my/work/latex/img/husiegel.png}
		\end{figure}
		\vspace*{-30mm}\hspace{-6mm}\textbf{\textcolor{pantone294}{\large{Mathematisch-Naturwissenschaftliche Fakultät}}}\\
		\textcolor{pantone294}{Institut für Physik}\\
		\vspace{20mm}
		\begin{center}
			%\textcolor{pantone294}{\huge{Grundpraktikum II}}\\\vspace*{7mm}
			\textcolor{pantone294}{\huge{\textbf{\thetitle}}}\\\vspace*{10mm}
			\textcolor{pantone294}{\thedate}\\\vspace*{10mm}
		\end{center}
		\tableofcontents

		\vspace{1cm}
 % summary

		Ein Abschnitt zum Zusammenfassen
	\end{titlepage}
	\makeatother
	\newpage

	\section{Eine kleine Einführung in GNU/Octave}

	\section{Paketbeschreibung}

	\begin{thebibliography}{}
	\bibitem{PraktikumsScript}Dr. Uwe Müller: \textit{Physikalisches Grundpraktikum: Einführung in die Messung, Auswertung und Darstellung experimenteller Ergebnisse in der Physik}, 2007
	\bibitem{Schaefer} Dr. P. Schäfer: \textit{Rechnet QtiPlot falsch?, oder Einige Anmerkungen zur Methode der linearen Regression}, 2017

%  \bibitem{PraktikumsScriptElektro}Dr. Uwe Müller: \textit{Physikalisches Grundpraktikum: Elektrodynamik und Optik}, 2010
%  \bibitem{datenanalyse} Prof. Dr. Volker Blobel, Prof. Dr. Erich Lohrmann: \textit{Statistische und numerische Methoden der Datenanalyse}, 1998
%  \bibitem{modern} Prof. Dr. Martin Erdmann, Prof. Dr. Thomas Hebbeker: \textit{Moderne Methoden der Datenanalyse}, 2013

	\end{thebibliography}
\end{document}
