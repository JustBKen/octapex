% Copyright 2019 Anton Douginets
%
% Licensed under the Apache License, Version 2.0 (the "License");
% you may not use this file except in compliance with the License.
% You may obtain a copy of the License at
%
%    http://www.apache.org/licenses/LICENSE-2.0
%
% Unless required by applicable law or agreed to in writing, software
% distributed under the License is distributed on an "AS IS" BASIS,
% WITHOUT WARRANTIES OR CONDITIONS OF ANY KIND, either express or implied.
% See the License for the specific language governing permissions and
% limitations under the License.

\documentclass[a4paper,11pt]{article}
\usepackage{ngerman}
\usepackage{soul}
\usepackage{mathtools}
\usepackage{amssymb,amsmath,amsfonts}
\usepackage[utf8]{inputenc}
\usepackage{graphicx}
\usepackage{geometry}
\usepackage{float}
\usepackage[autostyle=true,german=quotes]{csquotes}
\usepackage{gensymb}
\usepackage{units}
\usepackage{fancyhdr}
\usepackage[nottoc,numbib]{tocbibind}
\usepackage{cite}
\usepackage{abstract}
\usepackage{wrapfig}
\usepackage[font=small,labelfont=bf]{caption}
\graphicspath{ {figures/} }
\geometry{a4paper, left=25mm, right=25mm, top=20mm, bottom=20mm}

\pagestyle{plain}
%\pagestyle{fancy}
%\lhead{octapex}
%\rhead{Paketbeschreibung}
%\lfoot{left footer}
%\rfoot{right footer}

\title{Octave für das Grundpraktikum}
\author{Anton Douginets}
\date{\today}



\begin{document}
\maketitle
\thispagestyle{empty}

\begin{abstract}
Zusammenfassung
\end{abstract}

\tableofcontents
\newpage
\pagenumbering{arabic}

	\section{Eine kleine Einführung in GNU/Octave}

	\section{octapex Paketbeschreibung}

	\begin{thebibliography}{}
	\bibitem{PraktikumsScript}Dr. Uwe Müller: \textit{Physikalisches Grundpraktikum: Einführung in die Messung, Auswertung und Darstellung experimenteller Ergebnisse in der Physik}, 2007
	\bibitem{Schaefer} Dr. P. Schäfer: \textit{Rechnet QtiPlot falsch?, oder Einige Anmerkungen zur Methode der linearen Regression}, 2017

%  \bibitem{PraktikumsScriptElektro}Dr. Uwe Müller: \textit{Physikalisches Grundpraktikum: Elektrodynamik und Optik}, 2010
%  \bibitem{datenanalyse} Prof. Dr. Volker Blobel, Prof. Dr. Erich Lohrmann: \textit{Statistische und numerische Methoden der Datenanalyse}, 1998
%  \bibitem{modern} Prof. Dr. Martin Erdmann, Prof. Dr. Thomas Hebbeker: \textit{Moderne Methoden der Datenanalyse}, 2013

	\end{thebibliography}
\end{document}
